En el presente trabajo se estudiaran los bancos de filtros como herramientas para computar expansiones en series de señales en tiempo discreto. Para esto primero se expondrá brevemente que significa expandir una señal en series ortonormales y biortogonales. Luego se expondrán dos ejemplos de bancos de filtros utilizados para computar dos casos particulares de expansiones, que son la expansion de Haar y la expansion sinc. Con esto se verá la conexión entre los bancos de filtros y las expansiones en series de señales. Quedará claro de esta forma por que los bancos de filtros sirven para este fin. Finalemente, ademas de computar expansiones en series, tambien podemos aplicar el proceso inverso, o sea recuperar la señal original. Para esto se analizará el significado de reconstrucción perfecta en los bancos de filtros.
