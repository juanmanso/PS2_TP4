
	En el presente trabajo se estudiarán los bancos de filtros como herramientas para computar expansiones en series de señales en tiempo discreto. Para ello primero se expondrá brevemente qué significa expandir una señal en series ortonormales y biortogonales. Luego se expondrán dos ejemplos de bancos de filtros utilizados para computar dos casos particulares de expansiones, que son la expansión de Haar y la expansión sinc. Con ésto se verá la conexión entre los bancos de filtros y las expansiones en series de señales. Quedará claro de esta forma por qué los bancos de filtros sirven para este fin. Finalmente, además de computar expansiones en series, también se puede aplicar el proceso inverso, o sea recuperar la señal original. Para ésto se analizará el significado de reconstrucción perfecta en los bancos de filtros.
