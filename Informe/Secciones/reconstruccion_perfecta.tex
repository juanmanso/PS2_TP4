\paragraph{}
El objetivo de esta sección será aclarar el significado de reconstrucción perfecta. Analizaremos las implicaciones que tiene la reconstrucción perfecta en el dominio del tiempo, en el dominio modulado y en el dominio polifásico. Los análisis serán desarrollados para bancos de filtros de dos canales, pero se darán las fórmulas equivalentes para bancos genéricos de N canales.
\paragraph{}
El significado de reconstrucción perfecta, en el contexto de bancos de filtros, significa que la salida es una version retrasada y posiblemente escalada de la entrada:

\begin{equation}
	\hat{X}(z) = c z^{-k} X(z)
\end{equation}

Analizaremos el significado de esta definición en los distintos dominios antes dichos.

\subsection{Análisis en el Dominio del Tiempo}

Poner lo que dice de la pagina 130 a 133.

\begin{equation}
	\begin{pmatrix}
		\vdots \\
		y_0[0] \\
		y_1[0] \\
		y_0[1] \\
		y_1[1] \\
		\vdots \\
	\end{pmatrix}
	=
	\begin{pmatrix}
		\vdots \\
		X[0] \\
		X[1] \\
		X[2] \\
		X[3] \\
		\vdots \\
	\end{pmatrix}
	= T_a 
	\begin{pmatrix}
		\vdots \\
		x[0] \\
		x[1] \\
		x[2] \\
		x[3] \\
		\vdots \\
	\end{pmatrix}
\end{equation}

\begin{equation}
	T_a=
	\begin{bmatrix*}[c]
		\vdots & \vdots & \vdots & & \vdots & \vdots & \vdots\\
		h_0[L-1] & h_0[L-2] & h_0[L-3] & \hdots & h_0[0] & 0 & 0 \\
		h_1[L-1] & h_1[L-2] & h_1[L-3] & \hdots & h_1[0] & 0 & 0 \\
		0 & 0 & h_0[L-1] & \hdots & h_0[2] & h_0[1] & h_0[0] \\
		0 & 0 & h_1[L-1] & \hdots & h_1[2] & h_1[1] & h_1[0] \\
		\vdots & \vdots & \vdots & & \vdots & \vdots & \vdots \\
	\end{bmatrix*}
\end{equation}

Asumiendo que los filtros $h_i$ son FIR de largo $L=2K$, la matriz $T_a$ puede refactorizarse del siguiente modo:

\begin{equation}
	T_a=
	\begin{bmatrix*}[c]
		 & \vdots & \vdots &  & \vdots & \vdots & &\\
		\hdots & A_0 & A_1 & \hdots & A_{K-1} & 0 & \hdots  \\
		\hdots & 0 & A_0 & \hdots & A_{K-2} & A_{K-1} & \hdots  \\
		 & \vdots & \vdots &  & \vdots & \vdots & &\\		
	\end{bmatrix*}
\end{equation}

Siendo cada una de las submatrices $A_i$

\begin{equation}
	A_i=
	\begin{bmatrix*}[c]
		h_0[2K-1-2i] & h_0[2K-2-2i] \\
		h_1[2K-1-2i] & h_1[2K-2-2i] \\
	\end{bmatrix*}
\end{equation}

\subsection{Análisis en el Dominio Modulado}

Poner lo que dice de la pagina 134 a 136.

\subsection{Análisis en el Dominio Polifásico}

Poner lo que dice de la pagina 137 a 139.

\paragraph{}
La idea sería dar las formulas equivalentes para N canales. Estan en las páginas 184 a 188.
