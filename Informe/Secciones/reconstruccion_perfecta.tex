\paragraph{}
El objetivo de esta sección será aclarar el significado de reconstrucción perfecta. Analizaremos las implicaciones que tiene la reconstrucción perfecta en el dominio del tiempo, en el dominio modulado y en el dominio polifásico. Los análisis serán desarrollados para bancos de filtros de dos canales, pero se darán las fórmulas equivalentes para bancos genéricos de N canales.
\paragraph{}
El significado de reconstrucción perfecta, en el contexto de bancos de filtros, significa que la salida es una version retrasada y posiblemente escalada de la entrada:

\begin{equation}
	\hat{X}(z) = c z^{-k} X(z)
\end{equation}

Analizaremos el significado de esta definición en los distintos dominios antes dichos.

\subsection{Análisis en el Dominio del Tiempo}

Poner lo que dice de la pagina 130 a 133.

\subsection{Análisis en el Dominio Modulado}

Poner lo que dice de la pagina 134 a 136.

\subsection{Análisis en el Dominio Polifásico}

Poner lo que dice de la pagina 137 a 139.

\paragraph{}
La idea sería dar las formulas equivalentes para N canales. Estan en las páginas 184 a 188.
