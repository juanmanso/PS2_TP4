

	A lo largo del trabajo se presentaron herramientas para el entendimiento del concepto de reconstrucción perfecta, partiendo de los ejemplos de las expansiones de Haar y sinc. Tras esa introducción se realizó un análisis más minusioso donde, a partir de los conceptos de ortonormalidad y biortogonalidad, se generaliza para filtros de análisis $\vect{H}(z)$ y síntesis $\vect{G}(z)$. La prueba de la reconstrucción perfecta se realiza bajo los tres dominios: tiempo, modulado y polifásico; obteniéndose la misma solución. Finalmente se demuestra que para que la reconstrucción sea perfecta y libre de \emph{aliasing} el filtro de análisis $\vect{H}(z)$ debe ser lo más transparente posible, representando únicamente un retardo puro (y por consiguiente la transferencia debe ser un \emph{delay} puro).
