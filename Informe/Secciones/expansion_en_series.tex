\paragraph{}
En cuanto a expansión en series de señales en tiempo discreto se estudiarán dos tipos, las ortonormales y biortogonales. En todos los casos trabajaremos con señales $x[n] \in l_{2}(\mathbb{Z})$.

% ------------------------------------------------------------------------

\paragraph{Series Ortonormales}
En esta expansión se dispone de una base de funciones $\phi_k$ que satisfacen la condición de ortonormalidad:

\begin{equation}
	\langle \phi_k[n] , \phi_l[n] \rangle = \delta[k - l]
\end{equation}

\paragraph{}
Donde $\delta_[n]$ es la función delta de Kroenecker. Con esta base se puede expresar las señales $x[n]$ de la forma:

\begin{equation}
	x[n] = \sum_{z \in \mathbb{Z}} \langle \phi_k[l] , x[l] \rangle \phi_{k}[n] = \sum_{z \in \mathbb{Z}} X[k] \phi_{k}[n]
\end{equation}

\paragraph{}
Donde:

\begin{equation}
	X[k] = \langle \phi_k[l] , x[l] \rangle = \sum_{l} \phi_{k}^{*} x[l]
\end{equation}

Es la transformada de $x[n]$.

% ------------------------------------------------------------------------

\paragraph{Series Biortogonales}
En este tipo de expansión se dispone, no de una base ortonormal, sino de dos bases $\{ \phi_{k} \}$ y $\{ \tilde{\phi_{k}} \}$, que satisfacen la condición de biortogonalidad:

\begin{equation}
	\langle \phi_k[n] , \tilde{\phi_l}[n] \rangle = \delta[k - l]
\end{equation}

\paragraph{}
Condición que se verá que está relacionada con la construcción perfecta. De esta forma se puede expresar las señales $x[n]$ de la forma:

\begin{equation}
	x[n] = \sum_{z \in \mathbb{Z}} \langle \phi_k[l] , x[l] \rangle \tilde{\phi}_{k}[n] = \sum_{z \in \mathbb{Z}} \tilde{X}[k] \tilde{\phi}_{k}[n]
\end{equation}

\paragraph{}
O de manera equivalente:

\begin{equation}
	x[n] = \sum_{z \in \mathbb{Z}} \langle \tilde{\phi}_k[l] , x[l] \rangle \phi_{k}[n] = \sum_{z \in \mathbb{Z}} X[k] \phi_{k}[n]
\end{equation}

\paragraph{}
Donde:

\begin{equation}
	\tilde{X}[k] = \langle \phi_k[l] , x[l] \rangle
\end{equation}

\paragraph{}
Y:

\begin{equation}
	X[k] = \langle \tilde{\phi}_k[l] , x[l] \rangle
\end{equation}

\paragraph{}
Son las transformadas de $x[n]$, con respecto a las bases $\{ \phi_{k} \}$ y $\{ \tilde{\phi_{k}} \}$.
