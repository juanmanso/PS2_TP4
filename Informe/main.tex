\documentclass[10pt,a4paper]{article}

\usepackage[utf8]{inputenc}		% Configuro la codificación
\input{.command.tex}
% En el siguiente archivo se configuran las variables del trabajo práctico
%% \providecommand es similar a \newcommnad, salvo que el primero ante un 
%% conflicto en la compilación, es ignorado.

% Al comienzo de un TP se debe modificar los argumentos de los comandos

\providecommand{\myTitle}{TRABAJO PRÁCTICO 4} 
\providecommand{\mySubtitle}{Banco de filtros}

\providecommand{\mySubject}{Procesamiento de Señales II (86.52)}
\providecommand{\myKeywords}{UBA, Ingeniería, PS2}

\providecommand{\myAuthorSurname}{Anastopulos, Gasparovic, Manso}
\providecommand{\myTimePeriod}{Año 2018 - 2\textsuperscript{do} Cuatrimestre}

% No es necesario modificar este %%%%%%%%%%%%%%
\providecommand{\myHeaderLogo}{header_fiuba}
%%%%%%%%%%%%%%%%%%%%%%%%%%%%%%%%%%%%%%%%%%%%%%%%

% Si se utilizan listings, definir el lenguaje aquí
\providecommand{\myLanguage}{matlab} 
% Crear los integrantes del TP con el comando \PutMember donde
%%		1) Apellido, Nombre
%%		2) Número de Padrón
%%		3) E-Mail
\providecommand{\MembersOnCover}[0]
{		
		\PutMember{Anastópulos, Matías}{95120}{matias.anas@gmail.com}
		\PutMember{Gasparovic, Emiliano}{96123}{emilianit2000@gmail.com}
		\PutMember{Manso, Juan} {96133} {juanmanso@gmail.com}
}

\providecommand{\myGroupNumber}{02}


\Pagebreaktrue		% Setea si hay un salto de página en la carátula
\Indextrue
\Siunitxtrue			% Si quiero utilizar el paquete, \siunixtrue. Si no \siunixfalse
\Todonotestrue		% Habilita/Deshabilita las To-Do Notes y las funciones \unsure, \change, \info, \improvement y \thiswillnotshow.
\Listingstrue
\Keywordsfalse
\Putgrouptrue		% Habilita/Deshabilita el \myGroup en los headers
\Videofalse
				% Archivo con los comandos globales como Título y autores
\input{preamble.tex}
\input{aux_functions.tex}		% Se proveen un conjunto de funciones extras

% Defino el path de los includegraphics
\graphicspath{{./Figuras/}}		% Directorio que contiene los graficos

% Defino el path para los input de .tex y de .eps
\makeatletter
\def\input@path{{./Figuras/}{./Secciones/}{./Cover_page/}}
\makeatother

% Defino el path del listings
\ifListings
%% Cambiar el nombre de la carpeta si se utilizan Listings
	\lstinputpath{{../Octave/}}
\fi

\definecolor{myred}{rgb}{0.5,0,0}
\definecolor{mygreen}{rgb}{0,0.5,0}

\renewcommand{\thesubsection}{\thesection.\alph{subsection}}

\begin{document}
		% Carátula (formal o simple,_formal o _simple respectivamente) con Resumen
		% incluido e Índice (si es necesario configurar en config.tex) del informe
		\input{cover_formal.tex}
	\setcounter{page}{1}

\begin{center}{\Large{\textbf{Enunciado}}}\end{center}
%	\input{I0.tex}

	
	El sistema a identificar puede verse en la Figura \ref{fig:auto}, se trata de un sistema de amortiguación de un automóvil. 

\vspace*{\fill}
\begin{figure}[H]
\centering
\includegraphics[scale=0.7]{auto.png}
\caption{Sistema Fisico.}
\label{fig:auto} 
\end{figure}
\vspace*{\fill}

	Para identificar el sistema se utiliza el modelo que se observa en la Figura \ref{fig:mbk}. Se trata de un sistema de masa, amortiguador y resorte, donde la entrada $x_{1}$ es el nivel del suelo, y la salida $x_{2}$ es la altura en la que se encuentra el chasis del automóvil. La idea es, mediante un filtro adaptativo, poder estimar las constantes $m$, $b$ y $k$ del sistema.

\vspace*{\fill}
\begin{figure}[H]
\centering
\includegraphics[scale=0.6]{mbk.png}
\caption{Modelo.}
\label{fig:mbk} 
\end{figure}
\vspace*{\fill}

	
	\section{Modelado y discretización}\label{sec:ej1}
	La ecuación del sistema de masa, resorte y amortiguador es la siguiente:

\begin{equation*}
	m \ddot{x}_{0}(t) + k (x_{0}(t) - x_{1}(t)) + b (\dot{x}_{0}(t) - \dot{x}_{1}(t)) = 0
\end{equation*}

Para poder discretizar el mismo se utilizaron las siguientes aproximaciones:

\begin{equation*}
	\dot{x}_{0} \approx \frac{x_{0}(n + 1) - x_{0}(n)}{\Delta T}
\end{equation*}

\begin{equation*}
	\ddot{x}_{0} \approx \frac{x_{0}(n + 1) - 2 x_{0}(n) + x_{0}(n - 1)}{\Delta T^2}
\end{equation*}

\begin{equation*}
	\dot{x}_{1} \approx \frac{x_{1}(n + 1) - x_{1}(n)}{\Delta T}
\end{equation*}
donde $\Delta T$ es el período de muestreo. Con todas las aproximaciones anteriores, y la ecuación del sistema en tiempo continuo, se llega a la siguiente ecuación en diferencias para el tiempo discreto.

\begin{equation*}
	x_{0}(n) = \frac{b \;\Delta T}{m + b \;\Delta T} x_{1}(n) + \frac{k \;\Delta T^2 - b \;\Delta T}{m + b \;\Delta T} x_{1}(n - 1) - \frac{k \;\Delta T^2 - 2 m - b \;\Delta T}{m + b \;\Delta T} x_{0}(n - 1) - \frac{m}{m + b \;\Delta T} x_{0}(n - 2)
\end{equation*}


		
	\section{Filtro adaptativo}\label{sec:ej2}
	
	Se presenta el esquema del filtro propuesto en la Figura \ref{fig:filtro_adaptativo}. 
	Ambos sistemas (el desconocido y el filtro adaptativo) se excitan con una señal de referencia $u(n)$ de ruido blanco gaussiano.
	Luego se computa la diferencia entre ambas salidas y se utiliza esta señal de error para el ajuste de los parámetros del filtro mediante algún algoritmo adaptativo.
	Cuando el filtro aprenda a comportarse igual que el sistema a identificar, el error será bajo, y los coeficientes del filtro podrán utilizarse como estimadores de los parámetros del sistema a identificar.

\vspace*{\fill}
\begin{figure}[H]
\centering
\includegraphics[scale=0.5]{filtro_adaptativo.png}
\caption{Filtro Adatptativo.}
\label{fig:filtro_adaptativo} 
\end{figure}
\vspace*{\fill}

Sin embargo ésto tiene un problema. La salida del filtro adaptativo se computa de la siguiente manera:
\begin{equation*}
	y(n) = w^T U(n)
\end{equation*}
donde $U(n) = [u(n) \> u(n - 1) \> ... \> u(n - M + 1)]^T$, $M$ es el orden del filtro y $w$ son los coeficientes del mismo.

	Ésto significa que el filtro adaptativo propuesto sólo sirve para estimar la respuesta en frecuencia de sistemas FIR y el sistema bajo análisis tiene la forma de un filtro IIR en tiempo discreto:

\begin{equation*}
		x_{0}(n) = b_{0} \> x_{1}(n) + b_{1} \> x_{1}(n - 1) + a_{0} \> x_{0}(n - 1) + a_{1} \> x_{0}(n - 2)
\end{equation*}

Una posible solución a esto es, en lugar de sólo introducir ruido blanco a la entrada, combinar el ruido blanco con algunos valores pasados de la salida:

\begin{equation*}
	U(n) = [x_{1}(n) \> x_{1}(n - 1) \> x_{0}(n - 1) \> x_{0}(n - 2)]^T
\end{equation*}
donde ahora se excita al sistema con $x_{1}$ ruido blanco gaussiano. Así los coeficientes $w$ ajustados del filtro podrán utilizarse como estimadores:

\begin{equation*}
	[w_{0} \> w_{1} \> w_{2} \> w_{3}]^T  = [\hat{b}_{0} \> \hat{b}_{1} \> \hat{a}_{0} \> \hat{a}_{1}]^T
\end{equation*}

Ahora bien, como estos coeficientes son función de las constantes $m$, $b$ y $k$, pueden estimarse las mismas. El único componente que falta es la elección del algoritmo adaptativo. Se propone en principio utilizar el algoritmo LMS, que obtiene los parámetros $w$ de la siguiente manera:

\begin{equation*}
	w(n) = w(n - 1) + \mu \> U(n) \> [d(n) - U(n)^* w(n - 1)]
\end{equation*}

donde $\mu$ es una constante de aprendizaje.


	\section{Implementación del filtro adaptativo}\label{sec:ej3}
			
	A continuación se presenta el \emph{script} con la implementación del algorítmo. 
	\lstinputlisting[linerange=Implementacion-fin]{ej_3.m}

	En la Figura \ref{fig:ej3} se expone la convergencia de los coeficientes en función de la cantidad de muestras (tiempo). Se puede ver que para más de 300 muestras, la estimación es correcta. Sin embargo al presentarse ruido en la medición de la salida, la estimación fluctúa alrededor del valor correcto. 
	\Juan{No estoy seguro que esté taaan bien}
	Si la incertidumbre de la medición es muy grande, la estimación tardará más en converger y oscilará en un rango mayor entorno al verdadero.

	\begin{figure}[h!]
		\centering
		\includegraphics[width=1.0\textwidth, trim = 0cm 0cm 0cm 0cm]{graf_ej3.pdf}
		\caption{Convergencia de los coeficientes a partir de la estimación LMS.}
		\label{fig:ej3}
	\end{figure}

	\lstinputlisting[linerange=Resultados- ]{ej_3.m}
	Los resultados de la simulación a través de la función \texttt{solve()} se exponen en la siguiente tabla: 

		\begin{table}[h!]
			\centering
			\begin{tabular}{ccc}
				\toprule
				$m$ (fija)	& $k$	& $b$\\
				\midrule
				5&\num{2.9849}&\num{1.9972}\\
				\bottomrule
			\end{tabular}
			\caption{Resultados de la estimación de los parámetros.}
			\label{tab:res_ej3}
		\end{table}

		\Juan{No sé qué poner ahí}
	Se puede ver que los errores son del orden del 1\% siendo este valor muy bajo.





	\pagebreak
	\section{Comportamiento ante corte del resorte}\label{sec:ej4}
		

	\begin{figure}[h!]
		\centering
		\includegraphics[width=0.8\textwidth]{graf_ej4.pdf}
		\caption{Convergencia de los parámetros si se corta el resorte con $\sigma_N = \SI{.01}{\m}$.}
		\label{fig:ej4}
	\end{figure}

	En la presente Sección se analiza el caso en que se corta el resorte en el medio de la simulación, es decir que la constante $k$ se anula. Las variaciones de los coeficientes se presentan las Figuras \ref{fig:ej4} y \ref{fig:ej4bis}. 
	Se propone que $k\neq0$ en particular \num{.1} porque de lo contrario la convergencia es muy lenta. Si se supone que el error es del orden de \SI{.01}{\m} se ve en la Figura \ref{fig:ej4} que los coeficientes no convergen a los valores correctos. En cambio si se supone incertidumbres del orden de \SI{0.001}{\m} converge mejor aunque lentamente (\ref{fig:ej4bis}). 

	\begin{figure}[h!]
		\centering
		\includegraphics[width=0.8\textwidth]{graf_ej4bis.pdf}
		\caption{Convergencia de los parámetros si se corta el resorte con $\sigma_N = \SI{.001}{\m}$.}
		\label{fig:ej4bis}
	\end{figure}


	
		\begin{table}[h!]
			\centering
			\begin{tabular}{cccc}
				\toprule
				&$m$ (fija)	& $k$	& $b$\\
				\midrule
				$\sigma_N=\num{.01}$&5&$\num{2.4955}$&$\num{2.0053}$\\
				$\sigma_N=\num{.001}$&5&$\num{0.2455}$&$\num{2.0007}$\\
				\bottomrule
			\end{tabular}
		\end{table}

		La estimación de $k$ difiere mucho del valor real cuando el ruido es alto. Este resultado coincide con la mala estimación de los coeficientes de la Figura \ref{fig:ej4}. Del mismo modo se ratifica la buena estimación cuando el ruido es bajo, obteniendose $k=\num{0.2}$ y una mejor convergencia en la Figura \ref{fig:ej4bis}.


	\pagebreak
	\section{Filtro \emph{NLMS}}\label{sec:ej5}
		
% Acá entran los puntos 5 y 6
	El algoritmo NLMS se diferencia del LMS en que la correción que se hace es unitaria, es decir que se normaliza la señal de referencia. Así se mejora la convergencia del filtro porque se hace más insensible a los cambios de amplitud de la señal de entrada. La ecuación del filtro NLMS resulta entonces: 

	\begin{equation*}
		w(n) = w(n - 1) + \mu \> \frac{U(n)}{\|U(n)\|} \> [d(n) - U(n)^* w(n - 1)]
	\end{equation*}

	
	En la Figura \ref{fig:ej5} se expone la convergencia de los coeficientes utilizando el algoritmo NLMS, alcanzandose finalmente los valores correctos.
	\begin{figure}[h!]
		\centering
		\includegraphics[width=1.0\textwidth]{graf_ej5.pdf}
		\caption{Convergencia de los coeficientes a partir de la estimación NLMS.}
		\label{fig:ej5}
	\end{figure}

	Se replica el caso de la Sección \ref{sec:ej4} donde se corta el resorte. Al igual que en dicha sección, se impone $k=0,1$ y se ve que cuando $\sigma_N=0,001$ (Figura \ref{fig:ej5_k01}) el algoritmo converge a los valores correctos para todos los coeficientes mientras que con $\sigma_N=0,01$ (Figura \ref{fig:ej5_k0}) no se cumple. 
	
	\begin{figure}[h!]
		\centering
		\includegraphics[width=1.0\textwidth]{graf_ej5_k0.pdf}
		\caption{Convergencia de los coeficientes al cortarse el resorte con $\sigma_N=0,001$.}
		\label{fig:ej5_k0}
	\end{figure}
	
	\begin{figure}[h!]
		\centering
		\includegraphics[width=1.0\textwidth]{graf_ej5_k0bis.pdf}
		\caption{Convergencia de los coeficientes al cortarse el resorte con $\sigma_N=0,01$.}
		\label{fig:ej5_k01}
	\end{figure}

	\pagebreak

	\begin{figure}[h!]
		\centering
		\includegraphics[width=1.0\textwidth]{graf_LMS_vs_NLMS.pdf}
		\caption{Comparación de las convergencias con LMS y NLMS.}
		\label{fig:ej5_comp}
	\end{figure}

	Para comparar los filtros LMS y NLMS se genera la Figura \ref{fig:ej5_comp} que superpone los resultados de ambas estimaciones. Se puede ver que la estimación NLMS converge con una curva más suave que la de LMS mientras que esta última es más ruidosa. La misma conclusión se puede obtener analizando la Figura \ref{fig:ej5_comp} que contempla el error entre la estimación y el valor verdadero para las últimas 300 muestras. Alli se observa que las estimaciones en NLMS tienen un error más estable que las del LMS que tiene una mayor oscilación.

	\begin{figure}[h!]
		\centering
		\includegraphics[width=1.0\textwidth]{graf_err_LMS_vs_NLMS.pdf}
		\caption{Comparación de los valores finales de convergencia.}
		\label{fig:ej5_err}
	\end{figure}


	En cuanto a los resultados de la estimación de los parámetros se obtienen los siguientes resultados para el algoritmo NLMS:
		\begin{table}[h!]
			\centering
			\begin{tabular}{cccc}
				\toprule
				&$m$ (fija)	& $k$	& $b$\\
				\midrule
				Sin corte de resorte&5&$\num{2.9950}$&$\num{2.0008}$\\
				$\sigma_N=\num{.01}$&5&$\num{2.4009}$&$\num{1.9980}$\\
				$\sigma_N=\num{.001}$&5&$\num{0.2236}$&$\num{1.9973}$\\
				\bottomrule
			\end{tabular}
		\end{table}

	Con respecto al sistema funcionando correctamente (sin corte) se ve que el algortimo NLMS estima mejor que el LMS teniendo errores del $0,1$\% versus 1\%. Sin embargo, para el caso del corte de resorte ambos filtros se comportan de manera similar. Ésto puede deberse a que los filtros son en escencia el mismo (salvo la normalización de la corrección) y no pueden lidiar con el problema ante un ruido grande.


	\section{Conclusiones}\label{sec:conclusiones}
		\input{6_conclusiones.tex}

	% \appendix
\end{document}
